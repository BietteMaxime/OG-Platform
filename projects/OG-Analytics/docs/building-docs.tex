The static documentation consists of javadocs plus documentation in
\verb|OG-Analytics/docs| written using Dexy.

\section{Prerequisites}

See \url{http://dexy.it/install} for instructions on installing Dexy.

You will need to install the Markdown python package.
\verb|pip install Markdown| or \verb|easy_install Markdown| is the easiest way.

\section{Obtaining Source Code}

To run the documentation, you will need to clone and build the OG-Platform repository.

\url{https://github.com/OpenGamma/OG-Platform/blob/master/README.txt}

\section{Generating JARs}

To clean, update and build JARs, run:

<< d['scripts/build-jars.sh|idio|l']['ant'] >>

from the OG-Platform directory.

\section{Building Javadocs}

To generate the javadocs, run:

<< d['scripts/build-javadocs.sh|idio|l']['ant'] >>

from the OG-Platform directory.

\section{Generating JSONdocs}

In order to let us get access to Java source code and metadata, we need to run
javadocs again using a custom Doclet which saves information in a JSON format.

<< d['scripts/build-jsondocs.sh|idio|l']['ant'] >>

Run this from the OG-Analytics directory, i.e. 1 level up from this docs directory.

This should generate a file named \verb|javadoc-data.json| in the
\verb|docs| directory.

\section{Running Dexy}

Before beginning work on Dexy, you should update the project source code, JARs,
javadocs and jsondocs by running the 3 steps listed above.

If you are running Dexy for the first time, you will need to run the \verb|dexy setup|
command first. It would be a good idea to work through the
\href{http://www.dexy.it/docs/tutorials/0-hello-world/}{Hello, World dexy tutorial}
and to read the \href{http://www.dexy.it/docs/guide/documents-dependencies-and-filters/}{Documents, Dependencies and Filters}
document to get you familiar with Dexy before proceeding.

The bash script located in \verb|scripts/run-dexy.sh| is the best way to run dexy.

<< d['scripts/run-dexy.sh|pyg|l'] >>

This script should be run from the docs directory, not the scripts directory.

If you have an apache server configured to serve out of the
docs directory, then you can navigate to
\url{http://localhost/<< OG_VERSION >>/analytics} to view
the generated files.

It is also possible to just run Dexy via:

<< d['scripts/run-dexy.sh|idio|l']['run-dexy'] >>

but by skipping the other steps you won't have things like stylesheets or
images in the correct relative locations for HTML files to render them
properly.

You will need to set VERSION first or fill in the value of the current
OpenGamma version (you can use \verb|dev| to refer to the latest development
version).

You can leave off the \verb|--output| argument. This argument tells Dexy to
skip some other reports it usually runs at the end. It will take a little
longer to run if you leave this out, but you will get some more information
about the Dexy run if you do.

When Dexy successfully completes its run, then the generated documents will be
placed in the output/ directory.

If something goes wrong, look for information in the stack trace or the
\verb|logs/dexy.log| file.

